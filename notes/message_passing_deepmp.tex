%% LyX 2.3.1-1 created this file.  For more info, see http://www.lyx.org/.
%% Do not edit unless you really know what you are doing.
\documentclass[english]{article}
\usepackage[T1]{fontenc}
\usepackage[latin9]{inputenc}
\usepackage{amsmath}
\usepackage{amssymb}

\makeatletter
%%%%%%%%%%%%%%%%%%%%%%%%%%%%%% User specified LaTeX commands.
%%%%% NEW MATH DEFINITIONS %%%%%

\usepackage{amsmath,amsfonts,bm}

% Mark sections of captions for referring to divisions of figures
\newcommand{\figleft}{{\em (Left)}}
\newcommand{\figcenter}{{\em (Center)}}
\newcommand{\figright}{{\em (Right)}}
\newcommand{\figtop}{{\em (Top)}}
\newcommand{\figbottom}{{\em (Bottom)}}
\newcommand{\captiona}{{\em (a)}}
\newcommand{\captionb}{{\em (b)}}
\newcommand{\captionc}{{\em (c)}}
\newcommand{\captiond}{{\em (d)}}

% Highlight a newly defined term
\newcommand{\newterm}[1]{{\bf #1}}


% Figure reference, lower-case.
\def\figref#1{figure~\ref{#1}}
% Figure reference, capital. For start of sentence
\def\Figref#1{Figure~\ref{#1}}
\def\twofigref#1#2{figures \ref{#1} and \ref{#2}}
\def\quadfigref#1#2#3#4{figures \ref{#1}, \ref{#2}, \ref{#3} and \ref{#4}}
% Section reference, lower-case.
\def\secref#1{section~\ref{#1}}
% Section reference, capital.
\def\Secref#1{Section~\ref{#1}}
% Reference to two sections.
\def\twosecrefs#1#2{sections \ref{#1} and \ref{#2}}
% Reference to three sections.
\def\secrefs#1#2#3{sections \ref{#1}, \ref{#2} and \ref{#3}}
% Reference to an equation, lower-case.
\def\eqref#1{equation~\ref{#1}}
% Reference to an equation, upper case
\def\Eqref#1{Equation~\ref{#1}}
% A raw reference to an equation---avoid using if possible
\def\plaineqref#1{\ref{#1}}
% Reference to a chapter, lower-case.
\def\chapref#1{chapter~\ref{#1}}
% Reference to an equation, upper case.
\def\Chapref#1{Chapter~\ref{#1}}
% Reference to a range of chapters
\def\rangechapref#1#2{chapters\ref{#1}--\ref{#2}}
% Reference to an algorithm, lower-case.
\def\algref#1{algorithm~\ref{#1}}
% Reference to an algorithm, upper case.
\def\Algref#1{Algorithm~\ref{#1}}
\def\twoalgref#1#2{algorithms \ref{#1} and \ref{#2}}
\def\Twoalgref#1#2{Algorithms \ref{#1} and \ref{#2}}
% Reference to a part, lower case
\def\partref#1{part~\ref{#1}}
% Reference to a part, upper case
\def\Partref#1{Part~\ref{#1}}
\def\twopartref#1#2{parts \ref{#1} and \ref{#2}}

\def\ceil#1{\lceil #1 \rceil}
\def\floor#1{\lfloor #1 \rfloor}
\def\1{\bm{1}}
\newcommand{\train}{\mathcal{D}}
\newcommand{\valid}{\mathcal{D_{\mathrm{valid}}}}
\newcommand{\test}{\mathcal{D_{\mathrm{test}}}}

\def\eps{{\epsilon}}


% Random variables
\def\reta{{\textnormal{$\eta$}}}
\def\ra{{\textnormal{a}}}
\def\rb{{\textnormal{b}}}
\def\rc{{\textnormal{c}}}
\def\rd{{\textnormal{d}}}
\def\re{{\textnormal{e}}}
\def\rf{{\textnormal{f}}}
\def\rg{{\textnormal{g}}}
\def\rh{{\textnormal{h}}}
\def\ri{{\textnormal{i}}}
\def\rj{{\textnormal{j}}}
\def\rk{{\textnormal{k}}}
\def\rl{{\textnormal{l}}}
% rm is already a command, just don't name any random variables m
\def\rn{{\textnormal{n}}}
\def\ro{{\textnormal{o}}}
\def\rp{{\textnormal{p}}}
\def\rq{{\textnormal{q}}}
\def\rr{{\textnormal{r}}}
\def\rs{{\textnormal{s}}}
\def\rt{{\textnormal{t}}}
\def\ru{{\textnormal{u}}}
\def\rv{{\textnormal{v}}}
\def\rw{{\textnormal{w}}}
\def\rx{{\textnormal{x}}}
\def\ry{{\textnormal{y}}}
\def\rz{{\textnormal{z}}}

% Random vectors
\def\rvepsilon{{\mathbf{\epsilon}}}
\def\rvtheta{{\mathbf{\theta}}}
\def\rva{{\mathbf{a}}}
\def\rvb{{\mathbf{b}}}
\def\rvc{{\mathbf{c}}}
\def\rvd{{\mathbf{d}}}
\def\rve{{\mathbf{e}}}
\def\rvf{{\mathbf{f}}}
\def\rvg{{\mathbf{g}}}
\def\rvh{{\mathbf{h}}}
\def\rvu{{\mathbf{i}}}
\def\rvj{{\mathbf{j}}}
\def\rvk{{\mathbf{k}}}
\def\rvl{{\mathbf{l}}}
\def\rvm{{\mathbf{m}}}
\def\rvn{{\mathbf{n}}}
\def\rvo{{\mathbf{o}}}
\def\rvp{{\mathbf{p}}}
\def\rvq{{\mathbf{q}}}
\def\rvr{{\mathbf{r}}}
\def\rvs{{\mathbf{s}}}
\def\rvt{{\mathbf{t}}}
\def\rvu{{\mathbf{u}}}
\def\rvv{{\mathbf{v}}}
\def\rvw{{\mathbf{w}}}
\def\rvx{{\mathbf{x}}}
\def\rvy{{\mathbf{y}}}
\def\rvz{{\mathbf{z}}}

% Elements of random vectors
\def\erva{{\textnormal{a}}}
\def\ervb{{\textnormal{b}}}
\def\ervc{{\textnormal{c}}}
\def\ervd{{\textnormal{d}}}
\def\erve{{\textnormal{e}}}
\def\ervf{{\textnormal{f}}}
\def\ervg{{\textnormal{g}}}
\def\ervh{{\textnormal{h}}}
\def\ervi{{\textnormal{i}}}
\def\ervj{{\textnormal{j}}}
\def\ervk{{\textnormal{k}}}
\def\ervl{{\textnormal{l}}}
\def\ervm{{\textnormal{m}}}
\def\ervn{{\textnormal{n}}}
\def\ervo{{\textnormal{o}}}
\def\ervp{{\textnormal{p}}}
\def\ervq{{\textnormal{q}}}
\def\ervr{{\textnormal{r}}}
\def\ervs{{\textnormal{s}}}
\def\ervt{{\textnormal{t}}}
\def\ervu{{\textnormal{u}}}
\def\ervv{{\textnormal{v}}}
\def\ervw{{\textnormal{w}}}
\def\ervx{{\textnormal{x}}}
\def\ervy{{\textnormal{y}}}
\def\ervz{{\textnormal{z}}}

% Random matrices
\def\rmA{{\mathbf{A}}}
\def\rmB{{\mathbf{B}}}
\def\rmC{{\mathbf{C}}}
\def\rmD{{\mathbf{D}}}
\def\rmE{{\mathbf{E}}}
\def\rmF{{\mathbf{F}}}
\def\rmG{{\mathbf{G}}}
\def\rmH{{\mathbf{H}}}
\def\rmI{{\mathbf{I}}}
\def\rmJ{{\mathbf{J}}}
\def\rmK{{\mathbf{K}}}
\def\rmL{{\mathbf{L}}}
\def\rmM{{\mathbf{M}}}
\def\rmN{{\mathbf{N}}}
\def\rmO{{\mathbf{O}}}
\def\rmP{{\mathbf{P}}}
\def\rmQ{{\mathbf{Q}}}
\def\rmR{{\mathbf{R}}}
\def\rmS{{\mathbf{S}}}
\def\rmT{{\mathbf{T}}}
\def\rmU{{\mathbf{U}}}
\def\rmV{{\mathbf{V}}}
\def\rmW{{\mathbf{W}}}
\def\rmX{{\mathbf{X}}}
\def\rmY{{\mathbf{Y}}}
\def\rmZ{{\mathbf{Z}}}

% Elements of random matrices
\def\ermA{{\textnormal{A}}}
\def\ermB{{\textnormal{B}}}
\def\ermC{{\textnormal{C}}}
\def\ermD{{\textnormal{D}}}
\def\ermE{{\textnormal{E}}}
\def\ermF{{\textnormal{F}}}
\def\ermG{{\textnormal{G}}}
\def\ermH{{\textnormal{H}}}
\def\ermI{{\textnormal{I}}}
\def\ermJ{{\textnormal{J}}}
\def\ermK{{\textnormal{K}}}
\def\ermL{{\textnormal{L}}}
\def\ermM{{\textnormal{M}}}
\def\ermN{{\textnormal{N}}}
\def\ermO{{\textnormal{O}}}
\def\ermP{{\textnormal{P}}}
\def\ermQ{{\textnormal{Q}}}
\def\ermR{{\textnormal{R}}}
\def\ermS{{\textnormal{S}}}
\def\ermT{{\textnormal{T}}}
\def\ermU{{\textnormal{U}}}
\def\ermV{{\textnormal{V}}}
\def\ermW{{\textnormal{W}}}
\def\ermX{{\textnormal{X}}}
\def\ermY{{\textnormal{Y}}}
\def\ermZ{{\textnormal{Z}}}

% Vectors
\def\vzero{{\bm{0}}}
\def\vone{{\bm{1}}}
\def\vmu{{\bm{\mu}}}
\def\vtheta{{\bm{\theta}}}
\def\va{{\bm{a}}}
\def\vb{{\bm{b}}}
\def\vc{{\bm{c}}}
\def\vd{{\bm{d}}}
\def\ve{{\bm{e}}}
\def\vf{{\bm{f}}}
\def\vg{{\bm{g}}}
\def\vh{{\bm{h}}}
\def\vi{{\bm{i}}}
\def\vj{{\bm{j}}}
\def\vk{{\bm{k}}}
\def\vl{{\bm{l}}}
\def\vm{{\bm{m}}}
\def\vn{{\bm{n}}}
\def\vo{{\bm{o}}}
\def\vp{{\bm{p}}}
\def\vq{{\bm{q}}}
\def\vr{{\bm{r}}}
\def\vs{{\bm{s}}}
\def\vt{{\bm{t}}}
\def\vu{{\bm{u}}}
\def\vv{{\bm{v}}}
\def\vw{{\bm{w}}}
\def\vx{{\bm{x}}}
%\def\vtau{{\bm{\tau}}}
\def\vy{{\bm{y}}}
\def\vz{{\bm{z}}}

% Elements of vectors
\def\evalpha{{\alpha}}
\def\evbeta{{\beta}}
\def\evepsilon{{\epsilon}}
\def\evlambda{{\lambda}}
\def\evomega{{\omega}}
\def\evmu{{\mu}}
\def\evpsi{{\psi}}
\def\evsigma{{\sigma}}
\def\evtheta{{\theta}}
\def\eva{{a}}
\def\evb{{b}}
\def\evc{{c}}
\def\evd{{d}}
\def\eve{{e}}
\def\evf{{f}}
\def\evg{{g}}
\def\evh{{h}}
\def\evi{{i}}
\def\evj{{j}}
\def\evk{{k}}
\def\evl{{l}}
\def\evm{{m}}
\def\evn{{n}}
\def\evo{{o}}
\def\evp{{p}}
\def\evq{{q}}
\def\evr{{r}}
\def\evs{{s}}
\def\evt{{t}}
\def\evu{{u}}
\def\evv{{v}}
\def\evw{{w}}
\def\evx{{x}}
\def\evy{{y}}
\def\evz{{z}}

% Matrix
\def\mA{{\bm{A}}}
\def\mB{{\bm{B}}}
\def\mC{{\bm{C}}}
\def\mD{{\bm{D}}}
\def\mE{{\bm{E}}}
\def\mF{{\bm{F}}}
\def\mG{{\bm{G}}}
\def\mH{{\bm{H}}}
\def\mI{{\bm{I}}}
\def\mJ{{\bm{J}}}
\def\mK{{\bm{K}}}
\def\mL{{\bm{L}}}
\def\mM{{\bm{M}}}
\def\mN{{\bm{N}}}
\def\mO{{\bm{O}}}
\def\mP{{\bm{P}}}
\def\mQ{{\bm{Q}}}
\def\mR{{\bm{R}}}
\def\mS{{\bm{S}}}
\def\mT{{\bm{T}}}
\def\mU{{\bm{U}}}
\def\mV{{\bm{V}}}
\def\mW{{\bm{W}}}
\def\mX{{\bm{X}}}
\def\mY{{\bm{Y}}}
\def\mZ{{\bm{Z}}}
\def\mBeta{{\bm{\beta}}}
\def\mPhi{{\bm{\Phi}}}
\def\mLambda{{\bm{\Lambda}}}
\def\mSigma{{\bm{\Sigma}}}

% Tensor
\DeclareMathAlphabet{\mathsfit}{\encodingdefault}{\sfdefault}{m}{sl}
\SetMathAlphabet{\mathsfit}{bold}{\encodingdefault}{\sfdefault}{bx}{n}
\newcommand{\tens}[1]{\bm{\mathsfit{#1}}}
\def\tA{{\tens{A}}}
\def\tB{{\tens{B}}}
\def\tC{{\tens{C}}}
\def\tD{{\tens{D}}}
\def\tE{{\tens{E}}}
\def\tF{{\tens{F}}}
\def\tG{{\tens{G}}}
\def\tH{{\tens{H}}}
\def\tI{{\tens{I}}}
\def\tJ{{\tens{J}}}
\def\tK{{\tens{K}}}
\def\tL{{\tens{L}}}
\def\tM{{\tens{M}}}
\def\tN{{\tens{N}}}
\def\tO{{\tens{O}}}
\def\tP{{\tens{P}}}
\def\tQ{{\tens{Q}}}
\def\tR{{\tens{R}}}
\def\tS{{\tens{S}}}
\def\tT{{\tens{T}}}
\def\tU{{\tens{U}}}
\def\tV{{\tens{V}}}
\def\tW{{\tens{W}}}
\def\tX{{\tens{X}}}
\def\tY{{\tens{Y}}}
\def\tZ{{\tens{Z}}}


% Graph
\def\gA{{\mathcal{A}}}
\def\gB{{\mathcal{B}}}
\def\gC{{\mathcal{C}}}
\def\gD{{\mathcal{D}}}
\def\gE{{\mathcal{E}}}
\def\gF{{\mathcal{F}}}
\def\gG{{\mathcal{G}}}
\def\gH{{\mathcal{H}}}
\def\gI{{\mathcal{I}}}
\def\gJ{{\mathcal{J}}}
\def\gK{{\mathcal{K}}}
\def\gL{{\mathcal{L}}}
\def\gM{{\mathcal{M}}}
\def\gN{{\mathcal{N}}}
\def\gO{{\mathcal{O}}}
\def\gP{{\mathcal{P}}}
\def\gQ{{\mathcal{Q}}}
\def\gR{{\mathcal{R}}}
\def\gS{{\mathcal{S}}}
\def\gT{{\mathcal{T}}}
\def\gU{{\mathcal{U}}}
\def\gV{{\mathcal{V}}}
\def\gW{{\mathcal{W}}}
\def\gX{{\mathcal{X}}}
\def\gY{{\mathcal{Y}}}
\def\gZ{{\mathcal{Z}}}

% Sets
\def\sA{{\mathbb{A}}}
\def\sB{{\mathbb{B}}}
\def\sC{{\mathbb{C}}}
\def\sD{{\mathbb{D}}}
% Don't use a set called E, because this would be the same as our symbol
% for expectation.
\def\sF{{\mathbb{F}}}
\def\sG{{\mathbb{G}}}
\def\sH{{\mathbb{H}}}
\def\sI{{\mathbb{I}}}
\def\sJ{{\mathbb{J}}}
\def\sK{{\mathbb{K}}}
\def\sL{{\mathbb{L}}}
\def\sM{{\mathbb{M}}}
\def\sN{{\mathbb{N}}}
\def\sO{{\mathbb{O}}}
\def\sP{{\mathbb{P}}}
\def\sQ{{\mathbb{Q}}}
\def\sR{{\mathbb{R}}}
\def\sS{{\mathbb{S}}}
\def\sT{{\mathbb{T}}}
\def\sU{{\mathbb{U}}}
\def\sV{{\mathbb{V}}}
\def\sW{{\mathbb{W}}}
\def\sX{{\mathbb{X}}}
\def\sY{{\mathbb{Y}}}
\def\sZ{{\mathbb{Z}}}

% Entries of a matrix
\def\emLambda{{\Lambda}}
\def\emA{{A}}
\def\emB{{B}}
\def\emC{{C}}
\def\emD{{D}}
\def\emE{{E}}
\def\emF{{F}}
\def\emG{{G}}
\def\emH{{H}}
\def\emI{{I}}
\def\emJ{{J}}
\def\emK{{K}}
\def\emL{{L}}
\def\emM{{M}}
\def\emN{{N}}
\def\emO{{O}}
\def\emP{{P}}
\def\emQ{{Q}}
\def\emR{{R}}
\def\emS{{S}}
\def\emT{{T}}
\def\emU{{U}}
\def\emV{{V}}
\def\emW{{W}}
\def\emX{{X}}
\def\emY{{Y}}
\def\emZ{{Z}}
\def\emSigma{{\Sigma}}

% entries of a tensor
% Same font as tensor, without \bm wrapper
\newcommand{\etens}[1]{\mathsfit{#1}}
\def\etLambda{{\etens{\Lambda}}}
\def\etA{{\etens{A}}}
\def\etB{{\etens{B}}}
\def\etC{{\etens{C}}}
\def\etD{{\etens{D}}}
\def\etE{{\etens{E}}}
\def\etF{{\etens{F}}}
\def\etG{{\etens{G}}}
\def\etH{{\etens{H}}}
\def\etI{{\etens{I}}}
\def\etJ{{\etens{J}}}
\def\etK{{\etens{K}}}
\def\etL{{\etens{L}}}
\def\etM{{\etens{M}}}
\def\etN{{\etens{N}}}
\def\etO{{\etens{O}}}
\def\etP{{\etens{P}}}
\def\etQ{{\etens{Q}}}
\def\etR{{\etens{R}}}
\def\etS{{\etens{S}}}
\def\etT{{\etens{T}}}
\def\etU{{\etens{U}}}
\def\etV{{\etens{V}}}
\def\etW{{\etens{W}}}
\def\etX{{\etens{X}}}
\def\etY{{\etens{Y}}}
\def\etZ{{\etens{Z}}}

% The true underlying data generating distribution
\newcommand{\pdata}{p_{\rm{data}}}
% The empirical distribution defined by the training set
\newcommand{\ptrain}{\hat{p}_{\rm{data}}}
\newcommand{\Ptrain}{\hat{P}_{\rm{data}}}
% The model distribution
\newcommand{\pmodel}{p_{\rm{model}}}
\newcommand{\Pmodel}{P_{\rm{model}}}
\newcommand{\ptildemodel}{\tilde{p}_{\rm{model}}}
% Stochastic autoencoder distributions
\newcommand{\pencode}{p_{\rm{encoder}}}
\newcommand{\pdecode}{p_{\rm{decoder}}}
\newcommand{\precons}{p_{\rm{reconstruct}}}

\newcommand{\laplace}{\mathrm{Laplace}} % Laplace distribution

\newcommand{\E}{\mathbb{E}}
\newcommand{\Ls}{\mathcal{L}}
\newcommand{\R}{\mathbb{R}}
\newcommand{\emp}{\tilde{p}}
\newcommand{\lr}{\alpha}
\newcommand{\reg}{\lambda}
\newcommand{\rect}{\mathrm{rectifier}}
\newcommand{\softmax}{\mathrm{softmax}}
\newcommand{\sigmoid}{\sigma}
\newcommand{\softplus}{\zeta}
\newcommand{\KL}{D_{\mathrm{KL}}}
\newcommand{\Var}{\mathrm{Var}}
\newcommand{\standarderror}{\mathrm{SE}}
\newcommand{\Cov}{\mathrm{Cov}}
% Wolfram Mathworld says $L^2$ is for function spaces and $\ell^2$ is for vectors
% But then they seem to use $L^2$ for vectors throughout the site, and so does
% wikipedia.
\newcommand{\normlzero}{L^0}
\newcommand{\normlone}{L^1}
\newcommand{\normltwo}{L^2}
\newcommand{\normlp}{L^p}
\newcommand{\normmax}{L^\infty}

\newcommand{\parents}{Pa} % See usage in notation.tex. Chosen to match Daphne's book.

\DeclareMathOperator*{\argmax}{arg\,max}
\DeclareMathOperator*{\argmin}{arg\,min}
\DeclareMathOperator{\Proj}{Proj}
\DeclareMathOperator{\Approx}{Approx}
\DeclareMathOperator{\sign}{sign}
\DeclareMathOperator{\Tr}{Tr}
\let\ab\allowbreak

\makeatother

\usepackage{babel}
\begin{document}
\title{DeepMP derivation}
\maketitle

\section{Derivation of the BP and AMP equations}

\subsection{Derivation of the BP equations\label{sec:bp_derivation}}

In order to derive the BP equations, we start with the following portion
of the factor graph reported in Eq.~\ref{eq:factors} in the main
text, describing the contribution of a single data example in the
inner loop of the PasP updates:

\begin{equation}
\prod_{\ell=0}^{L}\prod_{k}\,P^{\ell+1}\left(x_{kn}^{\ell+1}\ \bigg|\ \sum_{i}W_{ki}^{\ell}x_{in}^{\ell}\right)\quad\text{where }\vx_{n}^{0}=\vx_{n},\ \vx_{n}^{L+1}=y_{n}.
\end{equation}
where we recall that the quantity $x_{kn}^{\ell}$ corresponds to
the activation of neuron $k$ in layer $\ell$ in correspondence of
the input example $n$. 

Let us start by analyzing the single factor:

\begin{equation}
P^{\ell+1}\left(x_{kn}^{\ell+1}\ \bigg|\ \sum_{i}W_{ki}^{\ell}x_{in}^{\ell}\right)
\end{equation}
We refer to messages that travel from input to output in the factor
graph as \textit{upgoing} or \textit{upwards} messages, while to the
ones that travel from output to input as \textit{downgoing} or \textit{backwards}
messages.

\paragraph{Factor-to-variable-$\mathbf{W}$ messages}

The factor-to-variable-$W$ messages read:

\begin{align}
\hat{\nu}_{kn\to ki}^{\ell+1}(W_{ki}^{\ell})\propto & \int\prod_{i'\neq i}d\nu_{ki'\to n}^{\ell}(W_{ki'}^{\ell})\prod_{i'}d\nu_{i'n\to k}^{\ell}(x_{i'n}^{\ell})\ d\nu_{\downarrow}(x_{kn}^{\ell+1})\ P^{\ell+1}\left(x_{kn}^{\ell+1}\ \bigg|\ \sum_{i'}W_{ki'}^{\ell}x_{i'n}^{\ell}\right)\label{eq:factor_to_W}
\end{align}
where $\nu_{\downarrow}$ denotes the messages travelling downwards
(from output to input) in the factor graph.

We denote the means and variances of the incoming messages respectively
with $m_{ki\to n}^{\ell},\,\hat{x}_{in\to k}^{\ell}$ and $\sigma_{ki\to n}^{\ell},\,\Delta_{in\to k}^{\ell}$:

\begin{align}
m_{ki\to n}^{\ell}= & \int d\nu_{ki\to n}^{\ell}(W_{ki}^{\ell})\ W_{ki}^{\ell}\\
\sigma_{ki\to n}^{\ell}= & \int d\nu_{ki\to n}^{\ell}(W_{ki}^{\ell})\ \left(W_{ki}^{\ell}-m_{ki\to n}^{\ell}\right)^{2}\\
\hat{x}_{in\to k}^{\ell}= & \int d\nu_{in\to k}^{\ell}(x_{in}^{\ell})\ x_{in}^{\ell}\\
\Delta_{in\to k}^{\ell}= & \int d\nu_{in\to k}^{\ell}(x_{in}^{\ell})\ \left(x_{in}^{\ell}-\hat{x}_{in\to k}^{\ell}\right)^{2}
\end{align}
We now use the central limit theorem to observe that with respect
to the incoming messages distributions - assuming independence of
these messages - in the large input limit the preactivation is a Gaussian
random variable:

\begin{equation}
\sum_{i'\neq i}W_{ki'}^{\ell}x_{i'n}^{\ell}\sim\mathcal{N}(\omega_{kn\to i}^{\ell},V_{kn\to i}^{\ell})
\end{equation}
where:

\begin{align}
\omega_{kn\to i}^{\ell} & =\mathbb{E}_{\nu}\left[\sum_{i'\neq i}W_{ki'}^{\ell}x_{i'n}^{\ell}\right]=\sum_{i'\neq i}m_{ki'\to n}^{\ell}\,\hat{x}_{i'n\to k}^{\ell}\\
V_{kn\to i}^{\ell} & =Var_{\nu}\left[\sum_{i'\neq i}W_{ki'}^{\ell}x_{i'n}^{\ell}\right]\nonumber \\
 & =\sum_{i'\neq i}\left(\sigma_{ki'\to n}^{\ell}\,\Delta_{i'n\to k}^{\ell}+\left(m_{ki'\to n}^{\ell}\right)^{2}\,\Delta_{i'n\to k}^{\ell}+\sigma_{ki'\to n}^{\ell}\,\left(\hat{x}_{i'n\to k}^{\ell}\right)^{2}\right)
\end{align}
Therefore we can rewrite the outgoing messages as:

\begin{equation}
\hat{\nu}_{kn\to i}^{\ell+1}(W_{ki}^{\ell})\propto\int dz\,d\nu_{in\to k}^{\ell}(x_{in}^{\ell})\ d\nu_{\downarrow}(x_{kn}^{\ell+1})\ e^{-\frac{\left(z-\omega_{kn\to i}-W_{ki}^{\ell}x_{in}^{\ell}\right)^{2}}{2V_{kn\to i}}}\,P^{\ell+1}\left(x_{kn}^{\ell+1}\ \bigg|\ z\right)
\end{equation}
We now assume $W_{ki}^{\ell}x_{in}^{\ell}$ to be small compared to
the other terms. With a second order Taylor expansion we obtain:

\begin{align}
\hat{\nu}_{kn\to i}^{\ell}(W_{ki}^{\ell})\propto & \int dz\,\ d\nu_{\downarrow}(x_{kn}^{\ell+1})\ e^{-\frac{\left(z-\omega_{kn\to i}\right)^{2}}{2V_{kn\to i}}}P^{\ell+1}\left(x_{kn}^{\ell+1}\ \bigg|\ z\right)\nonumber \\
 & \times\left(1+\frac{z-\omega_{kn\to i}}{V_{kn\to i}}\hat{x}_{in\to k}^{\ell}W_{ki}^{\ell}+\frac{\left(z-\omega_{kn\to i}\right)^{2}-V_{kn\to i}}{2V_{kn\to i}}\left(\Delta+\left(\hat{x}_{in\to k}^{\ell}\right)^{2}\right)\left(W_{ki}^{\ell}\right)^{2}\right)
\end{align}
Introducing now the function:

\begin{equation}
\varphi^{\ell}(B,A,\omega,V)=\log\int\mathrm{d}x\,\mathrm{d}z\ e^{-\frac{1}{2}Ax^{2}+Bx}\,P^{\ell}\left(x|z\right)e^{-\frac{\left(\omega-z\right)^{2}}{2V}}
\end{equation}
and defining:

\begin{align}
g_{kn\to i}^{\ell} & =\partial_{\omega}\varphi^{\ell+1}(B^{\ell+1},A^{\ell+1},\omega_{kn\to i}^{\ell},V_{kn\to i}^{\ell})\\
\Gamma_{kn\to i}^{\ell} & =-\partial_{\omega}^{2}\varphi^{\ell+1}(B^{\ell+1},A^{\ell+1},\omega_{kn\to i}^{\ell},V_{kn\to i}^{\ell})
\end{align}
the expansion for the log-message reads:

\begin{align}
\log\hat{\nu}_{kn\to i}^{\ell}(W_{ki}^{\ell}) & \approx const+\hat{x}_{in\to k}^{\ell}\,g_{kn\to i}^{\ell}W_{ki}^{\ell}\nonumber \\
 & -\frac{1}{2}\left(\left(\Delta_{in\to k}^{\ell}+\left(\hat{x}_{in\to k}^{\ell}\right)^{2}\right)\Gamma_{kn\to i}^{\ell}-\Delta_{in\to k}^{\ell}\left(g_{kn\to i}^{\ell}\right)^{2}\right)\left(W_{ki}^{\ell}\right)^{2}
\end{align}


\paragraph{Factor-to-variable-$\mathbf{x}$ messages}

The derivation of these messages is analogous to the factor-to-variable-$W$
ones in Eq.~\ref{eq:factor_to_W} just reported. The final result
for the log-message is:

\begin{align}
\log\hat{\nu}_{kn\to i}^{\ell}(x_{in}^{\ell})\approx & const+m_{ki\to n}^{\ell}\,g_{kn\to i}^{\ell}x_{in}^{\ell}\nonumber \\
 & -\frac{1}{2}\left(\left(\sigma_{ki\to n}^{\ell}+\left(m_{ki\to n}^{\ell}\right)^{2}\right)\Gamma_{kn\to i}^{\ell}-\sigma_{ki\to n}^{\ell}\left(g_{kn\to i}^{\ell}\right)^{2}\right)\left(x_{in}^{\ell}\right)^{2}
\end{align}


\paragraph{Variable-$\mathbf{W}$-to-output factor messages}

The message from variable $W_{ki}^{\ell}$ to the output factor $kn$
reads:

\begin{align}
\nu_{ki\to n}^{\ell}(W_{ki}^{\ell}) & \propto P_{\theta_{ki}}^{\ell}(W_{ki}^{\ell})e^{\sum_{n'\neq n}\log\hat{\nu}_{kn'\to i}^{\ell}(W_{ki}^{\ell})}\nonumber \\
 & \approx P_{\theta_{ki}}^{\ell}(W_{ki}^{\ell})e^{H_{ki\to n}^{\ell}W_{ki}^{\ell}-\frac{1}{2}G_{ki\to n}^{\ell}\left(W_{ki}^{\ell}\right)^{2}}
\end{align}
where we have defined:

\begin{align}
H_{ki\to n}^{\ell} & =\sum_{n'\neq n}\hat{x}_{in'\to k}^{\ell}\,g_{kn'\to i}^{\ell}\\
G_{ki\to n}^{\ell} & =\sum_{n'\neq n}\left(\left(\Delta_{in'\to k}^{\ell}+\left(\hat{x}_{in'\to k}^{\ell}\right)^{2}\right)\Gamma_{kn'\to i}^{\ell}-\Delta_{in'\to k}^{\ell}\left(g_{kn'\to i}^{\ell}\right)^{2}\right)
\end{align}
Introducing now the effective free energy:

\begin{align}
\psi^{\ell}(H,G,\theta) & =\log\int\mathrm{d}W\,\ P_{\theta}^{\ell}\left(W\right)e^{HW-\frac{1}{2}GW^{2}}
\end{align}
{[}TODO: $\psi^{\ell}$ ha l'indice di layer $\ell$ mentre nelle
equazioni sul paper non ce l'ha{]} we can express the first two cumulants
of the message $\nu_{ki\to n}^{\ell}(W_{ki}^{\ell})$ as:

\begin{align}
m_{ki\to n}^{\ell} & =\partial_{H}\psi^{\ell}(H_{ki\to n}^{\ell},G_{ki\to n}^{\ell},\theta_{ki})\\
\sigma_{ki\to n}^{\ell} & =\partial_{H}^{2}\psi^{\ell}(H_{ki\to n}^{\ell},G_{ki\to n}^{\ell},\theta_{ki})
\end{align}


\paragraph{Variable-$\mathbf{x}$-to-input factor messages}

We can write the downgoing message as:

\begin{align}
\nu_{\downarrow}(x_{in}^{\ell}) & \propto e^{\sum_{k}\log\hat{\nu}_{kn\to i}^{\ell}(x_{in}^{\ell})}\nonumber \\
 & \approx e^{B_{in}^{\ell}x-\frac{1}{2}A_{in}^{\ell}x^{2}}
\end{align}
where:

\begin{align}
B_{in}^{\ell} & =\sum_{n}m_{ki\to n}^{\ell}\,g_{kn\to i}^{\ell}\\
A_{in}^{\ell} & =\sum_{n}\left(\left(\sigma_{ki\to n}^{\ell}+\left(m_{ki\to n}^{\ell}\right)^{2}\right)\Gamma_{kn\to i}^{\ell}-\sigma_{ki\to n}^{\ell}\left(g_{kn\to i}^{\ell+1}\right)^{2}\right)
\end{align}


\paragraph{Variable-$\mathbf{x}$-to-output factor messages}

By defining the following cavity quantities:

\begin{align}
B_{in\to k}^{\ell} & =B_{in\to k}^{\ell}-m_{ki\to n}^{\ell}\,g_{kn\to i}^{\ell}\\
A_{in\to k}^{\ell} & =A_{in\to k}^{\ell}-\left(\left(\sigma_{ki\to n}^{\ell}+\left(m_{ki\to n}^{\ell}\right)^{2}\right)\Gamma_{kn\to i}^{\ell}-\sigma_{ki\to n}^{\ell}\left(g_{kn\to i}^{\ell}\right)^{2}\right)
\end{align}
and the following non-cavity ones:

\begin{align}
\omega_{kn}^{\ell} & =\sum_{i}m_{ki\to n}^{\ell}\,\hat{x}_{in\to k}^{\ell}\\
V_{kn}^{\ell} & =\sum_{i}\left(\sigma_{ki\to n}^{\ell}\,\Delta_{in\to k}^{\ell}+\left(m_{ki\to n}^{\ell}\right)^{2}\,\Delta_{in\to k}^{\ell}+\sigma_{ki\to n}^{\ell}\,\left(\hat{x}_{i'n\to k}^{\ell}\right)^{2}\right)
\end{align}
we can express the first 2 cumulants of the upgoing messages as:

\begin{align}
\hat{x}_{in\to k}^{\ell} & =\partial_{B}\varphi^{\ell}(B_{in\to k}^{\ell},A_{in\to k}^{\ell},\omega_{in}^{\ell-1},V_{in}^{\ell-1})\\
\Delta_{in\to k}^{\ell} & =\partial_{B}^{2}\varphi^{\ell}(B_{in\to k}^{\ell},A_{in\to k}^{\ell},\omega_{in}^{\ell-1},V_{in}^{\ell-1})
\end{align}


\paragraph{Wrapping it up}

Additional but straightforward considerations are required for the
final input and output layers ($\ell=0$ and $\ell=L+1$ respectively),
since they do not receive messages from below and above respectively.
In the end, thanks to indipendence assumptions and the central limit
theorem that we used throughout the derivations, we arrive to a closed
set of equations involving the means and the variances (or otherwise
the corresponding natural parameters) of the messages. 

{[}TODO mention that we approximate cavity variances with non-cavity
ones{]} {[}TODO use non-cavity variances for BP equations{]}

Dividing the update equations in a \textit{forward} and \textit{backward}
pass, and ordering them using time indexes in such a way that we have
an efficient flow of information, we obtain the set of BP equations
presented in the main text Eqs.~(\ref{eq:upd-x}-\ref{eq:upd-H}) and
in the Appendix Eqs.~(\ref{eqa:upd-x}-\ref{eqa:upd-H}). We report
the complete set of the BP equations here (TODO forse non c'� bisogno,
magari solo l'inizializzazione):

\paragraph{Initialization}

At $\tau=0$:

\begin{align}
B_{in\to k}^{\ell,0} & =0\\
A_{in}^{\ell,0} & =0\\
H_{ki\to n}^{\ell,0} & =0\\
G_{ki}^{\ell,0} & =0
\end{align}


\paragraph{Forward Pass}

At each $\tau=1,\dots,\tau_{max}$, for $\ell=0,\dots,L$ (TODO check
$\ell$ range -> $\hat{x}_{in\to k}^{\ell,\tau}$ non dovrebbe partire
da $\ell=1$?):

\begin{align}
\hat{x}_{in\to k}^{\ell,\tau} & =\partial_{B}\varphi^{\ell}(B_{in\to k}^{\ell,\tau-1},A_{in}^{\ell,\tau-1},\omega_{in}^{\ell-1,\tau},V_{in}^{\ell-1,\tau})\\
\Delta_{in}^{\ell,\tau} & =\partial_{B}^{2}\varphi^{\ell}(B_{in}^{\ell,\tau-1},A_{in}^{\ell,\tau-1},\omega_{in}^{\ell-1,\tau},V_{in}^{\ell-1,\tau})\\
m_{ki\to n}^{\ell,\tau} & =\partial_{H}\psi^{\ell}(H_{ki\to n}^{\ell,\tau-1},G_{ki}^{\ell,\tau-1},\theta_{ki}^{\ell})\\
\sigma_{ki}^{\ell,\tau} & =\partial_{H}^{2}\psi^{\ell}(H_{ki}^{\ell,\tau-1},G_{ki}^{\ell,\tau-1},\theta_{ki}^{\ell})\\
V_{kn}^{\ell,\tau} & =\sum_{i}\left(\left(m_{ki}^{\ell,\tau}\right)^{2}\Delta_{in}^{\ell,\tau}+\sigma_{ki}^{\ell,\tau-1}\left(\hat{x}_{in}^{\ell,\tau}\right)^{2}+\sigma_{ki}^{\ell,\tau-1}\Delta_{in}^{\ell,\tau}\right)\\
\omega_{kn\to i}^{\ell,\tau} & =\sum_{i'\neq i}m_{ki'\to n}^{\ell,\tau}\,\hat{x}_{i'n\to k}^{\ell,\tau}
\end{align}


\paragraph{Backward Pass}

For $\tau=1,\dots,\tau_{max}$, for $\ell=L,\dots,0$ (TODO check
$\ell$ range):

\begin{align}
g_{kn\to i}^{\ell,\tau} & =\partial_{\omega}\varphi^{\ell+1}(B_{kn}^{\ell+1,\tau},A_{kn}^{\ell+1,\tau},\omega_{kn\to i}^{\ell,\tau},V_{kn}^{\ell,\tau})\\
\Gamma_{kn}^{\ell,\tau} & =-\partial_{\omega}^{2}\varphi^{\ell+1}(B_{kn}^{\ell+1,\tau},A_{kn}^{\ell+1,\tau},\omega_{kn}^{\ell,\tau},V_{kn}^{\ell,\tau})\\
A_{in}^{\ell,\tau} & =\sum_{k}\left(\left(\left(m_{ki}^{\ell,\tau}\right)^{2}+\sigma_{ki}^{\ell,\tau}\right)\Gamma_{kn}^{\ell,\tau}-\sigma_{ki}^{\ell,\tau}\left(g_{kn}^{\ell,\tau}\right)^{2}\right)\\
B_{in\to k}^{\ell,\tau} & =\sum_{k'\neq k}m_{k'i\to n}^{\ell,\tau}\,g_{k'n\to i}^{\ell,\tau}\\
G_{ki}^{\ell,\tau} & =\sum_{n}\left(\left(\left(\hat{x}_{in}^{\ell,\tau}\right)^{2}+\Delta_{in}^{\ell,\tau}\right)\Gamma_{kn}^{\ell,\tau}-\Delta_{in}^{\ell,\tau}\left(g_{kn}^{\ell,\tau}\right)^{2}\right)\\
H_{ki\to n}^{\ell,\tau} & =\sum_{n'\neq n}\hat{x}_{in'\to k}^{\ell,\tau}\,g_{kn'\to i}^{\ell,\tau}
\end{align}


\subsection{Derivation of the AMP equations\label{sec:amp_derivation}}

In order to obtain the AMP equations, we approximate cavity quantities
with non-cavity ones in the BP equations Eqs.~(\ref{eqa:upd-x}-\ref{eqa:upd-H})
using a first order expansion. We start with the mean activation:

\begin{align}
\hat{x}_{in\to k}^{\ell,\tau}= & \partial_{B}\varphi^{\ell}(B_{in}^{\ell,\tau-1}-m_{ki\to n}^{\ell,\tau-1}\,g_{kn\to i}^{\ell,\tau-1},A_{in}^{\ell,\tau-1},\omega_{in}^{\ell-1,\tau},V_{in}^{\ell-1,\tau})\nonumber \\
\approx & \partial_{B}\varphi^{\ell}(B_{in}^{\ell,\tau-1},A_{in}^{\ell,\tau-1},\omega_{in}^{\ell-1,\tau},V_{in}^{\ell-1,\tau})\nonumber \\
 & -m_{ki\to n}^{\ell,\tau-1}\,g_{kn\to i}^{\ell,\tau-1}\partial_{B}^{2}\varphi^{\ell}(B_{in}^{\ell,\tau-1},A_{in}^{\ell,\tau-1},\omega_{in}^{\ell-1,\tau},V_{in}^{\ell-1,\tau})\nonumber \\
\approx & \hat{x}_{in}^{\ell,\tau}-m_{ki}^{\ell,\tau-1}g_{kn}^{\ell,\tau-1}\Delta_{in}^{\ell,\tau}
\end{align}
Analogously, for the weight's mean we have:

\begin{align}
m_{ki\to n}^{\ell,\tau} & =\partial_{H}\psi^{\ell}(H_{ki}^{\ell,\tau-1}-\hat{x}_{in\to k}^{\ell,\tau-1}\,g_{kn\to i}^{\ell,\tau-1},G_{ki}^{\ell,\tau-1},\theta_{ki}^{\ell})\nonumber \\
 & \approx\partial_{H}\psi^{\ell}(H_{ki}^{\ell,\tau-1},G_{ki}^{\ell,\tau-1},\theta_{ki}^{\ell})-\hat{x}_{in\to k}^{\ell,\tau-1}\,g_{kn\to i}^{\ell,\tau-1}\partial_{H}^{2}\psi^{\ell}(H_{ki}^{\ell,\tau-1},G_{ki}^{\ell,\tau-1},\theta_{ki}^{\ell})\nonumber \\
 & \approx m_{ki}^{\ell,\tau}-\hat{x}_{in}^{\ell,\tau-1}\,g_{kn}^{\ell,\tau-1}\,\sigma_{ki}^{\ell,\tau}.
\end{align}
These expansions bring us to the following equation for $\omega_{kn}^{\ell,\tau}$:

\begin{align}
\omega_{kn}^{\ell,\tau}= & \sum_{i}m_{ki\to n}^{\ell,\tau}\,\hat{x}_{in\to k}^{\ell,\tau}\nonumber \\
\approx & \sum_{i}m_{ki}^{\ell,\tau}\,\hat{x}_{in}^{\ell,\tau}-g_{kn}^{\ell,\tau-1}\sum_{i}\,\sigma_{ki}^{\ell,\tau}\hat{x}_{in}^{\ell,\tau}\hat{x}_{in}^{\ell,\tau-1}-g_{kn}^{\ell,\tau-1}\sum_{i}m_{ki}^{\ell,\tau}m_{ki}^{\ell,\tau-1}\Delta_{in}^{\ell,\tau}\nonumber \\
 & [maybe-irrelevant]+(g_{kn}^{\ell,\tau-1})^{2}\sum_{i}\sigma_{ki}^{\ell,\tau}m_{ki}^{\ell,\tau-1}\hat{x}_{in}^{\ell,\tau-1}\Delta_{in}^{\ell,\tau}
\end{align}
Let us now apply the same procedure to the other set of cavity messages.
A first order expansion of the quantities $g_{kn\to i}^{\ell,\tau}$
brings to:

\begin{align}
g_{kn\to i}^{\ell,\tau}= & \partial_{\omega}\varphi^{\ell+1}(B_{kn}^{\ell+1,\tau},A_{kn}^{\ell+1,\tau},\omega_{kn}^{\ell,\tau}-m_{ki\to n}^{\ell,\tau}\,\hat{x}_{in\to k}^{\ell,\tau},V_{kn}^{\ell,\tau})\nonumber \\
\approx & \partial_{\omega}\varphi^{\ell+1}(B_{kn}^{\ell+1,\tau},A_{kn}^{\ell+1,\tau},\omega_{kn}^{\ell,\tau},V_{kn}^{\ell,\tau})\nonumber \\
 & -m_{ki\to n}^{\ell,\tau}\,\hat{x}_{in\to k}^{\ell,\tau}\partial_{\omega}^{2}\varphi^{\ell+1}(B_{kn}^{\ell+1,\tau},A_{kn}^{\ell+1,\tau},\omega_{kn}^{\ell,\tau},V_{kn}^{\ell,\tau})\nonumber \\
\approx & g_{kn}^{\ell,\tau}+m_{ki}^{\ell,\tau}\hat{x}_{in}^{\ell,\tau}\Gamma_{kn}^{\ell,\tau}
\end{align}
and this expansion brings us to the following equations for $B_{in}^{\ell,\tau}$
and $H_{ki}^{\ell,\tau}$:

\begin{align}
B_{in}^{\ell,\tau}= & \sum_{k}m_{ki\to n}^{\ell,\tau}\,g_{kn\to i}^{\ell,\tau}\nonumber \\
\approx & \sum_{k}m_{ki}^{\ell,\tau}\,g_{kn}^{\ell,\tau}-\hat{x}_{in}^{\ell,\tau-1}\sum_{k}g_{kn}^{\ell,\tau}g_{kn}^{\ell,\tau-1}\sigma_{ki}^{\ell,\tau}+\hat{x}_{in}^{\ell,\tau}\sum_{k}(m_{ki}^{\ell,\tau})^{2}\Gamma_{kn}^{\ell,\tau}\nonumber \\
 & [maybe-irrelevant]-\hat{x}_{in}^{\ell,\tau}\hat{x}_{in}^{\ell,\tau-1}\sum_{k}\sigma_{ki}^{\ell,\tau}m_{ki}^{\ell,\tau}g_{kn}^{\ell,\tau-1}\Gamma_{kn}^{\ell,\tau}
\end{align}

\begin{align}
H_{ki}^{\ell,\tau}= & \sum_{n}\hat{x}_{in\to k}^{\ell,\tau}\,g_{kn\to i}^{\ell,\tau}\nonumber \\
\approx & \sum_{n}\hat{x}_{in}^{\ell,\tau}\,g_{kn}^{\ell,\tau}+m_{ki}^{\ell,\tau}\sum_{n}\left(\hat{x}_{in}^{\ell,\tau}\right)^{2}\Gamma_{kn}^{\ell,\tau}-m_{ki}^{\ell,\tau-1}\sum_{n}g_{kn}^{\ell,\tau}g_{kn}^{\ell,\tau-1}\Delta_{in}^{\ell,\tau}\nonumber \\
 & [maybe-irrelevant]-m_{ki}^{\ell,\tau}m_{ki}^{\ell,\tau-1}\sum_{n}g_{kn}^{\ell,\tau-1}\Gamma_{kn}^{\ell,\tau}\Delta_{in}^{\ell,\tau}\hat{x}_{in}^{\ell,\tau}
\end{align}
We are now able to write down the full AMP equations:

\paragraph{Initialization}

At $\tau=0$:

\begin{align}
B_{in}^{\ell,0} & =0\\
A_{in}^{\ell,0} & =0\\
H_{ki}^{\ell,0} & =0\text{ or some values}\\
G_{ki}^{\ell,0} & =0\text{ or some values}\\
g_{kn}^{\ell,0} & =0
\end{align}


\paragraph{Forward Pass}

At each $\tau=1,\dots,\tau_{max}$, for $\ell=0,\dots,L$ (TODO check
$\ell$ range):

\begin{align}
\hat{x}_{in}^{\ell,\tau}= & \partial_{B}\varphi^{\ell}(B_{in}^{\ell,\tau-1},A_{in}^{\ell,\tau-1},\omega_{in}^{\ell-1,\tau},V_{in}^{\ell-1,\tau})\\
\Delta_{in}^{\ell,\tau}= & \partial_{B}^{2}\varphi^{\ell}(B_{in}^{\ell,\tau-1},A_{in}^{\ell,\tau-1},\omega_{in}^{\ell-1,\tau},V_{in}^{\ell-1,\tau})\\
m_{ki}^{\ell,\tau}= & \partial_{H}\psi^{\ell}(H_{ki}^{\ell,\tau-1},G_{ki}^{\ell,\tau-1},\theta_{ki}^{\ell})\\
\sigma_{ki}^{\ell,\tau}= & \partial_{H}^{2}\psi^{\ell}(H_{ki}^{\ell,\tau-1},G_{ki}^{\ell,\tau-1},\theta_{ki}^{\ell})\\
V_{kn}^{\ell,\tau}= & \sum_{i}\left(\left(m_{ki}^{\ell,\tau}\right)^{2}\,\Delta_{in}^{\ell,\tau}+\sigma_{ki}^{\ell,\tau}\,\left(\hat{x}_{in}^{\ell,\tau}\right)^{2}+\sigma_{ki}^{\ell,\tau}\,\Delta_{in}^{\ell,\tau}\right)\\
\omega_{kn}^{\ell,\tau}= & \sum_{i}m_{ki}^{\ell,\tau}\,\hat{x}_{in}^{\ell,\tau}-g_{kn}^{\ell,\tau-1}\sum_{i}\,\sigma_{ki}^{\ell,\tau}\hat{x}_{in}^{\ell,\tau}\hat{x}_{in}^{\ell,\tau-1}-g_{kn}^{\ell,\tau-1}\sum_{i}m_{ki}^{\ell,\tau}m_{ki}^{\ell,\tau-1}\Delta_{in}^{\ell,\tau}\nonumber \\
 & +(g_{kn}^{\ell,\tau-1})^{2}\sum_{i}\sigma_{ki}^{\ell,\tau}m_{ki}^{\ell,\tau-1}\hat{x}_{in}^{\ell,\tau-1}\Delta_{in}^{\ell,\tau}
\end{align}


\paragraph{Backward Pass}

\begin{align}
g_{kn}^{\ell,\tau}= & \partial_{\omega}\varphi^{\ell+1}(B_{kn}^{\ell+1,\tau},A_{kn}^{\ell+1,\tau},\omega_{kn}^{\ell,\tau},V_{kn}^{\ell,\tau})\\
\Gamma_{kn}^{\ell,\tau}= & -\partial_{\omega}^{2}\varphi^{\ell+1}(B_{kn}^{\ell+1,\tau},A_{kn}^{\ell+1,\tau},\omega_{kn}^{\ell,\tau},V_{kn}^{\ell,\tau})\\
A_{in}^{\ell,\tau}= & \sum_{k}\left(\left(\left(m_{ki}^{\ell,\tau}\right)^{2}+\sigma_{ki}^{\ell,\tau}\right)\Gamma_{kn}^{\ell,\tau}-\sigma_{ki}^{\ell,\tau}\left(g_{kn}^{\ell,\tau}\right)^{2}\right)\\
B_{in}^{\ell,\tau}= & \sum_{k}m_{ki}^{\ell,\tau}\,g_{kn}^{\ell,\tau}-\hat{x}_{in}^{\ell,\tau-1}\sum_{k}g_{kn}^{\ell,\tau}g_{kn}^{\ell,\tau-1}\sigma_{ki}^{\ell,\tau}+\hat{x}_{in}^{\ell,\tau}\sum_{k}(m_{ki}^{\ell,\tau})^{2}\Gamma_{kn}^{\ell,\tau}\nonumber \\
 & -\hat{x}_{in}^{\ell,\tau}\hat{x}_{in}^{\ell,\tau-1}\sum_{k}\sigma_{ki}^{\ell,\tau}m_{ki}^{\ell,\tau}g_{kn}^{\ell,\tau-1}\Gamma_{kn}^{\ell,\tau}\\
G_{ki}^{\ell,\tau}= & \sum_{n}\left(\left(\left(\hat{x}_{in}^{\ell,\tau}\right)^{2}+\Delta_{in}^{\ell,\tau}\right)\Gamma_{kn}^{\ell,\tau}-\Delta_{in}^{\ell,\tau}\left(g_{kn}^{\ell,\tau}\right)^{2}\right)\\
H_{ki}^{\ell,\tau}= & \sum_{n}\hat{x}_{in}^{\ell,\tau}\,g_{kn}^{\ell,\tau}+m_{ki}^{\ell,\tau}\sum_{n}\left(\hat{x}_{in}^{\ell,\tau}\right)^{2}\Gamma_{kn}^{\ell,\tau}-m_{ki}^{\ell,\tau-1}\sum_{n}g_{kn}^{\ell,\tau}g_{kn}^{\ell,\tau-1}\Delta_{in}^{\ell,\tau}\nonumber \\
 & -m_{ki}^{\ell,\tau}m_{ki}^{\ell,\tau-1}\sum_{n}g_{kn}^{\ell,\tau-1}\Gamma_{kn}^{\ell,\tau}\Delta_{in}^{\ell,\tau}\hat{x}_{in}^{\ell,\tau}
\end{align}

TODO

- B and H equations have to be changed yet on the cl/amp branch in
DeepMP.jl

- not clear how to put damping

- presentare in appendice una sola volta la forma finale delle equazioni
di BP e AMP

- nei vari algoritmi riportati nel paper non viene specificata l'inizializzazione
dei messaggi
\end{document}
